\documentclass[12pt]{article}
\usepackage[utf8]{inputenc}
\usepackage[T1]{fontenc}
\usepackage[italian]{babel}

\usepackage{fancyhdr} % Per gli headers personalizzati
\usepackage{extramarks} % Per gli headers e i footers

\usepackage{amsmath}

\usepackage{multicol}
\setlength{\columnsep}{-20mm} % spazio tra le colonne- negativo per fare la seconda più grande della prima

\usepackage{listings} % per creare sezione dove scrivere ascii art
\lstdefinestyle{asciistyle}{ % definisco lo stile della sezione lst
    basicstyle=\ttfamily\footnotesize,
    breakatwhitespace=false,
    breaklines=true,
    keepspaces=true,
    showspaces=false,
    showstringspaces=false,
    showtabs=false,
    tabsize=4
}
\lstset{style=asciistyle} % setto lo stile della sezione lst come definito sopra

%\usepackage[showframe, pass]{geometry} % per vedere le dimensioni dei margini

% Margini
\topmargin=-10mm
\evensidemargin=10mm
\oddsidemargin=-10mm
\textwidth=180mm
\textheight=240mm
\headsep=5mm 

\headheight=10mm

% --------------------------------------------------------------------
% Definizione intestazione (non modificare!)
% --------------------------------------------------------------------
\newcommand{\HRule}[1]{\rule{\linewidth}{#1}} 	% Riga orizzontale

\makeatletter							% Titolo
\def\printtitle{%						
    {\centering \@title \par}}
\makeatother									

\makeatletter							% Ascii
\def\printascii{%						
    {\centering \large \@ascii}}
\makeatother

\makeatletter							% Autore
\def\printauthor{%					
    {\centering \large \@author}}				
\makeatother							

% --------------------------------------------------------------------
% Metadata (modificare questi)
% --------------------------------------------------------------------

\title{	\normalsize \textsc{Il ricettario del cibo semplice e buono} 	% Sottotitolo
		 	\\[1.5cm]								% 2cm spaziatura verticale
			\HRule{0.5pt} \\						% Riga orizzontale superiore
			\LARGE \textbf{\uppercase{Ctrl + Alt + Delizioso}}\\[0.2cm]	% Titolo
			\normalsize \textsc{Un ricettario in LaTeX}
			\HRule{0.5pt} \\ [0.5cm]		% Riga orizzontale inferiore + 0.5cm spaziatura verticale
			\normalsize \today			% Data ultima modifica (auto update)
			\\ [2cm]
		}

\author{
		Un libro di\\	
		Ivan aka ivndbt\\ [0.2cm]
		\scriptsize(Decisamente non un cuoco né uno scrittore)\\ [1cm]
		}


\begin{document}

% ------------------------------------------------------------------------------
% Intestazione (qui la stampo effettivamente)
% ------------------------------------------------------------------------------
\thispagestyle{empty}		% Non numerare la pagina del titolo

\printtitle					% Stampa il titolo come definito sopra
\begin{lstlisting}
										_ _  __
						*		*	   ( | )/_/		*		*
							*		__( >O< )  			*
									\_\(_|_)   


					            .-----.--,
					           (______(_. \ _
					            /     /   || |
					           :     /    || |
					           | .---------/|\-.      _.-----,
					           | :'-------'-'-':       |   = |-.
					 ___.--------:___          : _  _  |   = |-'
					'--.)________).--'______.-' (_)(_) :_____:


										_ _  __
							*		   ( | )/_/			*
						*		*	__( >O< )  		*		*
									\_\(_|_)   
	
\end{lstlisting}
  	\vfill
\printauthor				% Stampa l'autore come definito sopra
\newpage

% ------------------------------------------------------------------------------
% Header del documento
% ------------------------------------------------------------------------------
\pagestyle{fancy}

\rhead{ivndbt} % Header a destra
\lhead{Ctrl + Alt + Delizioso} % Header a sinistra
\chead{ } % Header centrale

% ------------------------------------------------------------------------------
% Pagina di introduzione e indice
% ------------------------------------------------------------------------------
\setcounter{page}{1}		% Inizia a numerare le pagine da qua
\section*{Introduzione}
Pronti per un'avventura deliziosa nel mondo di "Ctrl + Alt + Delizioso"? 
Questo libro presenta le mie ricette per la preparazione di cibo turbo mega, 
mirando a un equilibrio tra facilità di preparazione e sapore genuino. 

Sono piatti semplici, pensati per chi, come me, ama i sapori delicati e senza fronzoli. 
Le istruzioni sono dettagliate, ma vi invito con entusiasmo a sperimentare variazioni. 

Compilate il menu, avviate i fornelli e lanciate il comando \texttt{git gnam update ---recursive} !

\medskip
NDR: Questo documento è un lavoro in continuo aggiornamento a cadenza casuale.


\renewcommand{\contentsname}{Scegli la tua abbuffata}	% Ridefinisco il titolo dell'indice
\tableofcontents			% Faccio l'indice
\newpage

% ------------------------------------------------------------------------------
% Ricette
% ------------------------------------------------------------------------------
\section{Piadina}

\subsection*{Note, quantità e operazioni preliminari}
Per 2 persone.

\bigskip
\bigskip

\begin{multicols}{2}
\subsection*{Ingredienti}
\begin{itemize}
	\item 70 ml di acqua
    \item 2 cucchiai di olio extravergine di \\ oliva
    \item 1/2 cucchiaino di sale
    \item 140 gr di farina bianca 00
\end{itemize}

\vspace*{\fill}


\columnbreak
\subsection*{Procedimento}

Versare l'acqua a temperatura ambiente in una ciotola, aggiungere l'olio, il sale e mescolare.
\medskip

Aggiungere la farina e lavorare velocemente l'impasto con le mani, fino a formare una pallina.
\medskip

Mettere nel frigo la ciotola chiusa con pellicola e lasciar riposare 30 minuti.
\medskip

Dividere la pallina in 2 metà, stendere l'impasto e cuocere da entrambi i lati per almeno 3 minuti a lato.

\end{multicols}

\newpage

% ------------------------------------------------------------------------------

\section{Pizza}

\subsection*{Note, quantità e operazioni preliminari}
Per una teglia piccola.

\bigskip
\bigskip

\begin{multicols}{2}
\subsection*{Ingredienti}
\begin{itemize}
	\item 300 gr di farina bianca 00
	\item 13 gr di lievito madre in polvere
	\item 1 cucchiaino di sale
	\item 1/2 cucchiaio di zucchero
	\item 200 ml di acqua
\end{itemize}

\vspace*{\fill}

\columnbreak
\subsection*{Procedimento}

Mettere farina, lievito, sale, zucchero in una ciotola e mescolare il tutto.
Aggiungere, in un colpo solo, tutta l'acqua a temperatura ambiente e mescolare con una forchetta.
\medskip

Modellare l'impasto con le mani aggiungendo farina fino a che non smette di essere appiccicoso.
Formare una palla, cospargela di olio nella ciotola e coprire il tutto con pellicola trasparente.
Lasciare a lievitare in un luogo buio e asciutto (es. forno) per 4/5 ore.
\medskip

Rilavorare l'impasto per sonfiarlo e rimettere in forno il tutto.
Attendere 2 ore.
\medskip

Rilavorare l'impasto per sgonfiarlo e rimettere nuovamente il tutto in forno.
Attendere 2 ore.
\medskip

Cospargere la teglia di olio e stendere l'impasto.
In questa fase non bisogna preoccuparsi di riempire completamente la
superficie della teglia disonibile.
Lasciare in forno e attendere 1 ora.
\medskip

Togliere la teglia dal forno, accenderlo e portarlo a 230 °C in modalità ventilata.
Dare un ultima stesura all'impasto per colmare la teglia. Condire con solo pomodoro e olio.
Infornare per 7 minuti.
\medskip

Togliere la teglie e rapidamente aggiungere mozzarella e altri condimenti a piacere.
Infornare per 10 minuti circa (tenere controllato finché la crosta non ha preso colore).

\end{multicols}

\newpage

% ------------------------------------------------------------------------------

\section{Maionese}

\subsection*{Note, quantità e operazioni preliminari}
Tenere l'uovo fuori dal frigo per acclimatarlo.

\bigskip
\bigskip

\begin{multicols}{2}
\subsection*{Ingredienti}
\begin{itemize}
	\item 1 uovo a temperatura ambiente
	\item olio di semi
	\item sale
	\item pepe
	\item succo di limone
\end{itemize}

\vspace*{\fill}

\columnbreak
\subsection*{Procedimento}

Mettere nel recipiente del frullatore a immersione un dito di olio di semi
con una spolverata di sale e pepe.
\medskip

Aggiungere l'uovo (rosso e bianco) e iniziare a frullare.
In questa fase bisonga continuamente muovere la mano verso l'alto e verso il basso.
Contemporaneamente continuare a versare un filo d'olio nella salsa.
\medskip

Prima ch la maionese sia del tutto addensata, fermarsi.
Aggiungere un goccio di spremuta di limone e eventualmente aggiustare sale e pepe.
\medskip

Ricominciare a frullare continuando ad aggiungere olio come prima
fino al raggiungimento della consistenza desiderata.
\medskip

Conservare in frigorifero in uno scatolino ermeticamente chiuso.

\end{multicols}

\newpage

% ------------------------------------------------------------------------------

\section{Pasta alla carbonara}

\subsection*{Note, quantità e operazioni preliminari}
Per 2 persone.

\bigskip
\bigskip

\begin{multicols}{2}
\subsection*{Ingredienti}
\begin{itemize}
	\item 1 uovo
	\item 50 gr di pancetta
	\item olio
	\item pepe
	\item 140 gr di pasta
	\item sale grosso
	\item acqua
\end{itemize}

\vspace*{\fill}

\columnbreak
\subsection*{Procedimento}

Tagliare la pancetta a cubetti. Farla rosolare in padella con un filo d'olio.
Contemporaneamente mettere sul fuoco l'acqua della pasta.
Quando la pancetta è cotta, spegnere il fornello e continuare con la pasta.
\medskip

Circa due minuti prima che la pasta abbia finito la cottura: metto in una ciotola 
l'uovo intero e lo sbatto. Aggiungo pepe generosamente e un po' di grana grattuggiato.
\medskip

Riaccendo il fuoco della pancetta al minimo, scolo la pasta e la butto nella padella della 
pancetta mescolandole e rosolandole insieme per un momento.
\medskip

Con la padella calda e il fuoco spento: verso l'uovo e mescolo in tutto rapidamente
per evitare di cuocere l'uovo.
\medskip

Servire e aggiungere grana grattuggiato a piacimento.

\end{multicols}

\newpage


% ------------------------------------------------------------------------------

\section{Crepes}

\subsection*{Note, quantità e operazioni preliminari}
Per 2 persone.

\bigskip
\bigskip

\begin{multicols}{2}
\subsection*{Ingredienti}
\begin{itemize}
	\item 125 gr di farina
	\item 2 uova
	\item 330 ml di latte
	\item 20 gr di zucchero
	\item 10 gr di sale
\end{itemize}

\vspace*{\fill}

\columnbreak
\subsection*{Procedimento}

Mettere in una ciotola metà del latte preparato, uova (tuorlo e albume), sale, zucchero e farina.
\medskip

Mescolare con la frusta il tutto, aggiungendo (quando l'impasto si addensa) il latte restante.
\medskip

Se l'impasto risultasse troppo denso: aggiungere un po' d'acqua.
\medskip

Lasciare a riposare in frigo per mezz'ora prima di usarlo.
\medskip

Versare una porzione d'impasto al centro della padella e stenderlo con un cucchiaio con movimenti circolari. 
Attendere finché il colore non diventa marroncino/beige. 
Sollevare delicatamente con spatola e servire.

\end{multicols}

\newpage

% ------------------------------------------------------------------------------

\section{Omelette}

\subsection*{Note, quantità e operazioni preliminari}
Per 1 persona.

\bigskip
\bigskip

\begin{multicols}{2}
\subsection*{Ingredienti}
\begin{itemize}
	\item 1 uovo
	\item 4 pomodorini
	\item 1 sottiletta
	\item burro
	\item olio
	\item sale
	\item pepe
	\item basilico
	\item origano
\end{itemize}

\vspace*{\fill}

\columnbreak
\subsection*{Procedimento}

Tagliare in piccoli pezzi i pomodorini e il formaggio.
Metterli in una ciotola e condirli con olio, sale, pepe, basilico e origano.
\medskip

In un'altra ciotola, rompere l'uovo e sbatterlo con energia fino a montarlo (leggermente).
Aggiustare le quantità di sale e pepe.
\medskip

Far scaldare nel tegamino il burro, versare l'uovo e mettere il coperchio.
Senza mai girare l'impasto, cuocere a fuoco basso.
\medskip

Dopo un paio di minuti, disporre al centro i pomodorini e il formaggio.
Chiudere l'uovo a metà e rimettere il coperchio per un minuto ancora circa.
\medskip

Servire caldo.

\end{multicols}

\newpage

% ------------------------------------------------------------------------------

\section{Roast Beef}

\subsection*{Note, quantità e operazioni preliminari}
Un bel rotolone.

\bigskip
\bigskip

\begin{multicols}{2}
\subsection*{Ingredienti}
\begin{itemize}
	\item rotolo di roast beef
	\item farina
	\item pepe
	\item rosmarino
	\item salvia
	\item burro
	\item vino rosso o birra
	\item sale
\end{itemize}

\vspace*{\fill}

\columnbreak
\subsection*{Procedimento}

Accendere il forno a 180°C.
\medskip

Rovesciare in una fondina un'abbondante manciata di farina e
rotolarci la carne in blocco.
Mettere il rotolo in una pirofila.
A piacimento aggiungere pepe tritato,
salvia e rosmarino (probabilmente già legati con lo spago intorno al rotolo).
\medskip

Preparare del burro sciolto in un pentolino e rovesciarlo sul rotolo prima di infornarlo
per fare la crostina.
\medskip

Infornare per 10 minuti.
\medskip

Girare a testa in giù il rotolo e aggiungere vino rosso o birra.
Tenere girato per ulteriori 20 minuti in forno.
\medskip

Attendere raffreddamento per tagliare fettine più sottili.

\end{multicols}

\newpage

% ------------------------------------------------------------------------------

\section{Ragù}

\subsection*{Note, quantità e operazioni preliminari}
4 o 5 barattoli di marmellata piccoli.

\bigskip
\bigskip

\begin{multicols}{2}
\subsection*{Ingredienti}
\begin{itemize}
	\item 500gr vitello/maiale macinato
	\item 400gr passata di pomodoro
	\item 1/4 cipolla
	\item 1 carota
	\item rosmarino
	\item 2 bicchieri di brodo
\end{itemize}

\vspace*{\fill}

\columnbreak
\subsection*{Procedimento}

Se non disponibili preparare 2 bicchieri di brodo (utilizzando
mezzo dado in totale) nel microonde per 4 minuti.
\medskip

Nella pentola del minestrone far scaldare un dito di olio.
Aggiungere carota e cipolla tagliate a pezzettini piccoli.
Far rosolare a fuoco alto finché la cipolla non è dorata.
Aggiungere la carne.
Lasciare che la carne rosoli (diventi di color marroncino).
Aggiungere tutta la passata e i due bicchieri di brodo.
\medskip

Mescolare il tutto. Aggiungere rosmarino e pepe.
Portare a bollore.
\medskip

A questo punto abbassare il fuoco al minimo, 
mettere il coperchio e lasciare a risposo per 90 minuti.
\medskip

Scaduto il tempo verificare se il ragù è acquoso.
Se sì alzo la fiamma a coperchio aperto per farlo restringere.
Se no è pronto.

\end{multicols}

\newpage

% ------------------------------------------------------------------------------

\section{Besciamella}

\subsection*{Note, quantità e operazioni preliminari}
300gr. Per quantità differenti mantenere il rapporto 3:1:1 tra
latte, farina e burro.

\bigskip
\bigskip

\begin{multicols}{2}
\subsection*{Ingredienti}
\begin{itemize}
	\item 300gr latte intero
	\item 30gr farina
	\item 30gr burro
	\item 2 pizzichi di sale grosso
	\item noce moscata
\end{itemize}

\vspace*{\fill}

\columnbreak
\subsection*{Procedimento}

Mettere il latte nel bicchierone alto di plastica del
frullino. Scaldare in microonde per 4 minuti (finché
non bolle). Contemporaneamente nel pentolino far sciogliere
il burro a fuoco lento (senza farlo diventare scuro).
\medskip

Salare il latte e aggiungere un sospetto di noce moscata.
\medskip

Quando il burro sarà completamente sciolto, toglierlo dal fuoco,
aggiungere la farina e mescolare con forza con una frusta.
Ad amalgamazione completata, rimettere sul fuoco
(sempre al minimo) continuando a mescolare sempre, finché non
diventa di un bel giallo (quasi marrone).
\medskip

Aggiungere il latte ancora caldo nel pentolino poco per volta,
tenendo il fuoco al minimo e mescolando lentamente per amalgamare.
\medskip

Usare subito.

\end{multicols}

\newpage

% ------------------------------------------------------------------------------

\section{Lasagne}

\subsection*{Note, quantità e operazioni preliminari}
Una teglia piccola (3 porzioni).
Considerare 4 strati di condimento.

\bigskip
\bigskip

\begin{multicols}{2}
\subsection*{Ingredienti}
\begin{itemize}
	\item 3 barattoli ragù
	\item 300gr besciamella
	\item sfoglie della lasagna
	\item grana
	\item sottilette
	\item olio
\end{itemize}

\vspace*{\fill}

\columnbreak
\subsection*{Procedimento}

Spalmare un cucchiaio di besciamella sul fondo della teglia
per evitare che attacchi. Disporre il primo strato di sfoglia. 
Mettere 3 cucchiai di besciamella e 3 cucchiai abbondanti
di ragù. Posizionare una sottiletta divisa in 4 parti sul ragù.
Aggiungere una spolverata di grana grattuggiato.
\medskip

Ripetere il procedimento fino ad arrivare al quarto strato di 
sfoglia.
\medskip

Sopra il quarto strato di sfoglia, condire con besciamella,
ragù, sottiletta, abbondante grana grattuggiato e un filo di olio.
\medskip

Infornare a 190°C ventilato per 30 minuti.
\medskip

Togliere e gustare.

\end{multicols}

\newpage

% ------------------------------------------------------------------------------

\section{Casoncelli alla carbonara}

\subsection*{Note, quantità e operazioni preliminari}
Per 2 persone (indecise se bergamasche o romane).

\bigskip
\bigskip

\begin{multicols}{2}
\subsection*{Ingredienti}
\begin{itemize}
	\item 1 uovo
	\item 50 gr di pancetta
	\item olio
	\item pepe
	\item casoncelli
	\item sale grosso
	\item acqua
\end{itemize}

\vspace*{\fill}

\columnbreak
\subsection*{Procedimento}
*Il procedimento è identico a quello della pasta alla carbonara.
Volevo scrivere e condividere questa ricetta per rivendicare la sua ideazione.
\medskip

Provare prima di giudicare.
\medskip

Tagliare la pancetta a cubetti. Farla rosolare in padella con un filo d'olio.
Contemporaneamente mettere sul fuoco l'acqua.
Considerare che i casoncelli sono pronti in pochi minuti.
Buttarli in acqua quando la pancetta è a buon punto.
\medskip

Subito dopo aver buttato i casoncelli in acqua metto in una ciotola 
l'uovo intero e lo sbatto. Aggiungo pepe generosamente e un po' di grana grattuggiato.
\medskip

Abbasso il fuoco della pancetta al minimo, scolo i casoncelli e li butto nella padella della 
pancetta mescolando e rosolando insieme per un momento.
\medskip

Con la padella calda e il fuoco spento: verso l'uovo e mescolo in tutto rapidamente
per evitare di cuocere l'uovo.
\medskip

Servire e aggiungere grana grattuggiato a piacimento.

\end{multicols}

\newpage






\end{document}



